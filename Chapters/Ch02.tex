\chapter{Completions}
	\setlist[enumerate,1,2]{leftmargin=1.8cm}
	In this chapter we are intended to do \textit{elementary calculus} with a heavy taste of algebra instead of analysis. A lot of concepts in analysis are stolen by algebraists with different generalisations.
	\section{Absolute value}
		Let $K$ be a field. An \textbf{absolute value} on $K$ is a function $K \to \mathbb{R}$, $x \mapsto |x|_v$ satisfying the following three properties:
		\begin{enumerate}[start=1,label={\bfseries AV \arabic*}]
			\item We have $|x|_v \ge 0$ and $=0$ if and only if $x = 0$.
			\item For all $x,y \in K$ we have $|xy|_v = |x|_v|y|_v$.
			\item $|x+y|_v \le |x|_v + |y|_v$. \label{av3}
		\end{enumerate}
	
		If instead of \ref{av3} the absolute value satisfies the stronger condition
		\begin{enumerate}[start=4,label={\bfseries AV \arabic*}]
			\item $|x+y|_v \le \max(|x|_v,|y|_v)$
		\end{enumerate}
		then we shall say that it is a \textbf{valuation} or that it is non-archimedean. in particular, if the absolute value $|\cdot|$ satisfy $|x|=1$ for all $x \ne 0$, then we say $|\cdot|$ is \textbf{trivial}. We will exclude this case from now on. Our first goal is to study many possible valuations in general and in $\Q$. We are more interested in non-archimedean case because it \textit{breaks} our common sense in algebra:
		\begin{theorem}
			An absolute value $|\cdot|$ is non-archimedean if and only if there exists $M>0$ such that $|n \cdot 1|<M$ for all $n \in \N$.
		\end{theorem}
		\begin{proof}
			If $|\cdot|$ is non-archimedean, then
			\[
				|n| = |1+\cdots+1| \le 1, \forall n \in \N.
			\]
			
			Conversely, pick $x,y \in K$ and suppose $|x| \ge |y|$. Then
			\[
				\begin{aligned}
					|x+y|^n &= |(x+y)^n| \\
							&= \left|\sum_{j=0}^{n}{n \choose j}x^jy^{n-j}\right| \\
							&\le \sum_{j=0}^{n}\left|{n \choose j} \right||x|^j|y|^{n-j} \\
							&\le M(n+1)|x|^n.
				\end{aligned}
			\]
			Hence
			\[
				|x+y| \le \left(M(1+n)\right)^{1/n}|x| \to |x| \quad (n \to \infty).
			\]
		\end{proof}
		If $K$ is a infinite dimensional space, then this is not a big deal, but if we put $K=\Q$, does it make sense? Normally we would think the norm should grow bigger as $n$ grows.
	
	\section{Polynomials in complete fields}