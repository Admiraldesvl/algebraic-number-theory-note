\chapter{Completions}
	\setlist[enumerate,1,2]{leftmargin=1.8cm}
	In this chapter we are intended to do \textit{elementary calculus} with a heavy taste of algebra instead of analysis. A lot of concepts in analysis are stolen by algebraists with different generalisations.
	\section{Absolute value}
		Let $K$ be a field. An \textbf{absolute value} on $K$ is a function $K \to \mathbb{R}$, $x \mapsto |x|_v$ satisfying the following three properties:
		\begin{enumerate}[start=1,label={\bfseries AV \arabic*}]
			\item We have $|x|_v \ge 0$ and $=0$ if and only if $x = 0$.
			\item For all $x,y \in K$ we have $|xy|_v = |x|_v|y|_v$.
			\item $|x+y|_v \le |x|_v + |y|_v$. \label{av3}
		\end{enumerate}
	
		If instead of \ref{av3} the absolute value satisfies the stronger condition
		\begin{enumerate}[start=4,label={\bfseries AV \arabic*}]
			\item $|x+y|_v \le \max(|x|_v,|y|_v)$
		\end{enumerate}
		then we shall say that it is a \textbf{valuation} or that it is non-archimedean. in particular, if the absolute value $|\cdot|$ satisfy $|x|=1$ for all $x \ne 0$, then we say $|\cdot|$ is \textbf{trivial}. We will exclude this case from now on. 