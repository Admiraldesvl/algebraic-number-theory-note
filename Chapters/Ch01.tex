\chapter{Algebraic Integers}
	\section{Integral closure and algebraic numbers}
		Problems in solving polynomial equations give rise to a lot of concepts in algebra and geometry. If we are specifically interested in $\mds{Z}$, we have the concept of \textbf{algebraic integers}.
		
		\begin{definition}
			A finite extension $K$ of the rational number $\mds{Q}$ is called a \textbf{number field}. The integral closure of $\mds{Z}$ in $K$ is called the ring of \textbf{algebraic integers} of $K$, and is denoted by $\OK$. To be precise, every element $x \in \OK$ is a zero of a monic polynomial $f \in \Z[X].$
		\end{definition}
	
		For this concept we have a lot of classic examples:
		
		\begin{example}
			If $K=\Q$, then $\OK$ is simply $\Z$. This is intuitive because suppose $x=a/b \in \Q$ is integral over $\Z$ where $(a,b)=1$, then
			\[
				x^n+c_1x^{n-1}+\cdots+c_n = 0
			\]
			where $c_i \in \Z$. Multiplying by $b^n$ yields
			\[
				a^n+c_1a^{n-1}b+\cdots+c_nb = 0
			\]
			Hence $b$ divides $a^n$. But we also have $(a^n,b)=1$, hence $b=\pm 1$, which is to say $x \in \Z$.
			
			There is an more general setting. Since $\Z$ is a unique factorial domain (UFD), and UFD is integrally closed \href{https://proofwiki.org/wiki/Unique_Factorization_Domain_is_Integrally_Closed}{[proof]}, we have $\Z=\OK$. 
		\end{example}
	
		\begin{example}
			The Gaussian rational $\Q(i)=K$. Indeed it is natural to consider Gaussian integer $\Z[i]$ first. For any $z=m+ni \in \Z[i]$, we have
			\[
				z^2-2mz+m^2+n^2=0
			\]
			Hence $\Z[i] \subset \OK$. The converse is similar to our proof when $K=\Q$.
		\end{example}
	
		\begin{example}
			Cyclotomic field $K=Q(\xi_n)$, where $\xi_n$ is the root of $x^n-1$. In this case we have $\OK = \Z[\xi_n]$.
		\end{example}
		
		\begin{example}
			Quadratic field $K=\Q(\sqrt{d})$, where $d$ is a square-free integer $>1$. This time the algebraic integer ring is different from what you may have thought: $\OK = \Z[\omega]$ where
				\[
					\omega = \begin{cases}
						\frac{1+\sqrt{d}}{2}, &\quad d = 4k+1, \\
						\sqrt{d}, &\quad \text{otherwise}.
					\end{cases}
				\]
		\end{example}
		It turns out we are studying polynomials such as
		\begin{itemize}
			\item $x^2+1=0$.
			\item $x^n-1=0$.
			\item $x^2-d=0$.
		\end{itemize}
		It also turns out that many properties are not restricted to $\Z$, but to a specific class of rings. Hence we will investigate some properties in the sense of commutative ring theory. First we investigate separable extensions.
		
		\begin{theorem}
			Let $A$ be a Noetherian ring, integrally closed. Let $L$ be a finite separable extension of its quotient field $K$. Then the integral closure of $A$ in $L$ is finitely generated over $A$.
		\end{theorem}
		\begin{proof}
			Since $A$ is Noetherian, all submodules of a finitely generated module over $A$ is finitely generated. Hence it suffices to prove that the integral closure of $A$ is contained in a finitely generated $A$-module. % TODO: FINISH THE PROOF
		\end{proof}
		Note $Z$ is itself a Noetherian ring and integrally closed. $\Q$ is the fraction ring of $\Z$, and finite extensions of $\Q$ are always separable. It follows immediately that
		\begin{corollary}
			$\OK$ is finitely generated over $\Z$.
		\end{corollary}
	
		Next we study the degree of $\OK$.
		
		\begin{theorem}
			Let $A$ be a principal ideal ring, and $L$ a finite separable extension of its quotient field $K$, of degree $n$. Let $B$ be the integral closure of $A$ in $L$. Then $B$ is a free module of rank $n$ over $A$. 
		\end{theorem}
	
		\begin{proof}
			% TODO: FINISH THE PROOF.
		\end{proof}
	
		Next we study the localisation.
		
		\begin{theorem}
			Let $A$ be a subring of a ring $B$, which is integral over $A$. Let $S$ be a multiplicative subset of $A$. Then $S^{-1}B$ is integral over $S^{-1}A$. If $A$ is integrally closed, then $S^{-1}A$ is integrally closed.
		\end{theorem}
	
		\begin{proof}
			% TODO: FINISH THE PROOF.
		\end{proof}
	
		\begin{corollary}
			If $B$ is the integral closure of $A$ in some field extension $L$ of the quotient field of $A$, then $S^{-1}B$ is the integral closure of $S^{-1}A$ in $L$.
		\end{corollary}
		\begin{proof}
			% TODO: FINISH THE PROOF.
		\end{proof}